\documentclass{article}
\usepackage[top=1in, bottom=1in, left=1in, right=1in]{geometry}
\usepackage{url}
\usepackage{fancyhdr}
\pagestyle{fancy}
\usepackage{setspace}
\usepackage{graphicx}
\usepackage{verbatim}
\usepackage{listings}
\usepackage{xspace}
\usepackage[usenames,dvipsnames]{color}
\usepackage{tabularx}
\usepackage{multirow}

\newcommand{\us}{Joe Redmon and Doug Woos}
\newcommand{\class}{CSE 561 Project 1}

\newcommand{\todo}[1]{{{\color{blue}#1}}}

\title{\class} \author{\us}

\lhead{\class}
\rhead{\us}

%\newenvironment{outline}{}{}
\newenvironment{outline}{\comment}{\endcomment}

% macros
\begin{document}
\maketitle

\section{Implementation}
For modulation, We used an audio frequency shift keying (AFSK) scheme
modeled on that used by the Bell 202 modem. We used a two-tone scheme,
designating 1200Hz as the ``mark'' (indicating a 1) and 2400Hz as the
``space'' (indicating a 0). To decode the incoming audio data, we used
a simple zero-crossing algorithm, counting the number of times that
the waveform changes sign in order to determine its frequency over a
selected interval. We used a known preamble in order to determine the
offset at which to begin decoding.

We found that the transmission volume was critical in acheiving
acceptable error rates. We initially tried to transmit at the phone's
maximum volume, but found that it introduced far too much noise. We
reduced the volume until we could reliably find the preamble when the
phones were adjacent.

We experimented with a variety of symbols per bit, focusing on those
which yielded zero-crossing counts which were close to whole numbers
for the mark and space tones.

\section{Experiments}
We first tried to maximize our bitrate by reducing the symbols per
bit. When we held the phones directly adjacent to each other, we could
transmit with 37 symbols per bit, yielding a bitrate of 1192 bps with
an error rate of approximately 1~\%. However, we could not transmit
over any significant distance at this bitrate. For all subsequent
experiments, we transmitted 74 symbols per bit for a bitrate of 600
bps.

For the distance experiments, we transmitted 1KB of randomly generated
data (using a consistent seed for repeatability and error detection)
from one phone to the other while varying the distance. We repeated
each experiment 10 times, and we report both the average and the
standard deviation. Our results are in
Figure~\ref{fig:distance_results}.
\end{document}
